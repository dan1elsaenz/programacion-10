% Tipo de documento y otras especificaciones
\documentclass[12pt,letterpaper]{article}
% Configuracion
% Para escribir tildes y eñes
\usepackage[utf8]{inputenc}

% Para que los títulos de figuras, tablas y otros estén en español
\usepackage[spanish]{babel}

% Modificar el tamaño de section
\usepackage{titlesec}

% Los paquetes ams son desarrollados por la American Mathematical Society y mejoran la escritura de fórmulas y símbolos matemáticos.
\usepackage{amsmath}
\usepackage{amsfonts}
\usepackage{amssymb}

% Para insertar gráficas
\usepackage{graphicx}

% Para colocar varias figuras
\usepackage[lofdepth,lotdepth]{subfig}

% Para la presentación correcta de unidades
\usepackage{unitsdef}

% Redimensionamiento del espacio entre magnitud y unidad
\renewcommand{\unitvaluesep}{\hspace*{4pt}}

% Para insertar hipervínculos y marcadores
\usepackage[colorlinks=true,urlcolor=blue,linkcolor=black,citecolor=green]{hyperref}

% Para ubicar las tablas y figuras justo después del texto
\usepackage{float}

% Para hacer tablas más estilizadas
\usepackage{booktabs}

% Para manejar los encabezados y pies de página
\usepackage{fancyhdr}

% Cambiar nombre a tablas
\addto\captionsspanish{\renewcommand{\tablename}{Tabla}}
% Cambiar nombre a lista de tablas
\addto\captionsspanish{\renewcommand{\listtablename}{Índice de tablas}}

% Tamaño del área de escritura de la página
\usepackage[left=18mm,right=18mm,top=21mm,bottom=21mm]{geometry}

%Codigo
\usepackage{listings}

% Letra para el código
\usepackage{minted}

% Configuración para minted con fuente Inconsolata
\setminted{
  fontfamily=zi4, % inconsolata
  linenos,
  frame=lines,
  framesep=2mm,
  baselinestretch=1.2,
  fontsize=\footnotesize,
  breaklines=true,
  tabsize=4,
  numbersep=5pt
}


\usepackage{xcolor}

% Definicion de colores
\definecolor{codegreen}{rgb}{0,0.6,0}
\definecolor{codegray}{rgb}{0.5,0.5,0.5}
\definecolor{codepurple}{rgb}{0.58,0,0.82}
\definecolor{backcolour}{rgb}{0.95,0.95,0.92}
\definecolor{mygray}{rgb}{0.9,0.9,0.9}

\lstdefinestyle{mystyle}{
    language=Python, % Specify the programming language
    backgroundcolor=\color{white}, % Background color for the listing
    commentstyle=\color{codegreen}, % Comment style
    keywordstyle=\color{codepurple}, % Keyword style
    numberstyle=\tiny\color{codegray}, % Line number style
    stringstyle=\color{blue}, % String literal style
    basicstyle=\ttfamily\footnotesize, % Basic style
    breakatwhitespace=false, % Break lines only at whitespace
    breaklines=true, % Automatic line breaking
    captionpos=b, % Position of the caption
    keepspaces=true, % Keep spaces in the listing
    numbers=left, % Position of line numbers
    numbersep=5pt, % Padding between line numbers and code
    showspaces=false, % Show spaces with underscores?
    showstringspaces=false, % Show spaces in strings?
    showtabs=false, % Show tabs within strings?
    tabsize=2 % Tab size
}

\usepackage{lastpage}
\graphicspath{{./Figuras/}}
\pagenumbering{arabic}



% Contenido de los encabezados y pies de pagina
\pagestyle{fancy}
\lhead{Compendio Algoritmos}
\rhead{II Eliminatoria - OCI 2025}
\lfoot{Colegio Científico de Puriscal}
\cfoot{\thepage\ de \pageref{LastPage}}
\rfoot{Programación - Décimo}

\let\OLDthebibliography=\thebibliography
\def\thebibliography#1{\OLDthebibliography{#1}%
\addcontentsline{toc}{section}{\refname}}

\makeatletter
\renewcommand\@biblabel[1]{#1. \ }
\makeatother

% Section
\titleformat{\section}
  {\large\bfseries}
  {\thesection.}
  {1em} % Espacio entre número y título
  {}

% Subsection
\titleformat{\subsection}
  {\normalsize\bfseries}
  {\thesubsection}
  {1em}
  {}

% Espaciado
\titlespacing*{\section}{0pt}{1.5ex plus 1ex minus .2ex}{1ex}
\titlespacing*{\subsection}{0pt}{1ex plus .5ex minus .2ex}{0.8ex}

% Arreglar advertencia de márgenes
\setlength{\headheight}{14.49998pt}



\begin{document}

\begin{titlepage}

% Define un nuevo comando para las líneas horizontales, cambio de grosor
\newcommand{\HRule}{\rule{\linewidth}{0.5mm}}

% Centra todo en la página
\center

%----------------------------------------------------------------------------------------
%	SECCIONES DE ENCABEZADO
%----------------------------------------------------------------------------------------

% Nombre de la universidad
\textsc{\LARGE Colegio Científico de Puriscal}\\[1.5cm]
% Nombre de la escuela
\textsc{\Large Curso de Programación}\\[0.5cm]
% Título del curso
\textsc{\large Décimo año}\\[0.5cm]

%----------------------------------------------------------------------------------------
%	SECCIÓN DE TÍTULO
%----------------------------------------------------------------------------------------
\bigskip
\bigskip
\bigskip
\bigskip

% Título del documento
\HRule \\[0.4cm]
{ \huge \bfseries Compendio de Algoritmos\\[0.5cm] \LARGE II Eliminatoria OCI 2025}\\[0.4cm]
\HRule \\[1.5cm]

%----------------------------------------------------------------------------------------
%	SECCIÓN DE AUTOR
%----------------------------------------------------------------------------------------

\begin{center} \large
Prof. Daniel Alberto Sáenz Obando\\
\bigskip
\bigskip
\bigskip
\bigskip
\end{center}
\bigskip
\bigskip
\bigskip
\bigskip
\bigskip
\bigskip
\bigskip
\bigskip
\bigskip
\bigskip
\bigskip
\bigskip
\bigskip
\bigskip
\bigskip
\bigskip
\normalsize 6 de agosto 2025 \vspace*{5\baselineskip}
\vfill % Rellena el resto de la página con espacios en blanco

\end{titlepage}


\tableofcontents

\newpage

\section{Formato de Input y Output}

\subsection{Líneas separadas}

Sirve para una cantidad establecida de números en líneas separadas.

\begin{minted}{python}
cantidad1 = int(input())
cantidad2 = int(input())
\end{minted}

\subsection{Cantidad definida de números en una misma línea}

En el ejemplo se muestra la lectura de tres números a partir de un \texttt{input()}.
Se sabe previamente que se va a leer esa cantidad y se asigna cada uno a una variable con \texttt{map()}.

\begin{minted}{python}
n1, n2, n3 = map(int, input().split())
\end{minted}

Una alternativa es guardar los números leídos en una lista:

\begin{minted}{python}
# Forma 1
lista1 = list(map(int, input().split()))

# Forma 2
lista2 = [int(x) for x in input().split()]
\end{minted}

\subsection{Cantidad indefinida de números en una misma línea}

Se aplica \textit{list comprehension} para que se separe cada valor por espacio, a partir del \texttt{split()}. Luego, se convierten dichos strings en enteros con el \texttt{for} loop.

\begin{minted}{python}
numeros = [int(x) for x in input().split()]
\end{minted}

\subsection{Lectura completa del \texttt{stdin}}

Lee todo el \texttt{stdin} (entrada).
Para detener la lectura en caso de ejecutarse en un entorno interactivo como VS Code, utilizar el comando \texttt{Ctrl+Z} y luego \texttt{Enter}.

\begin{minted}{python}
import sys

lectura = sys.stdin.read()
\end{minted}

\subsection{Lectura de una cantidad determinada de líneas}

Suponga que se desea leer $n$ líneas de la terminal.
Para no utilizar \texttt{input()} muchas veces, se puede utilizar el siguiente comando.
Cada línea contiene \texttt{.split()}, en caso de que se desee separar su contenido interno por espacio (para varias palabras).

\begin{minted}{python}
# Forma 1
lineas = [input().split() for _ in range(n)]

# Forma 2
lista = []
for _ in range(n):
    lista.append(input().split())
\end{minted}

\subsection{Lectura de un tablero}

Note que la lectura de las columnas corresponde a la lista interna.
Mientras que, la lectura de las filas corresponde a la externa.
Se utiliza el valor leído de las filas para saber cuántas leer, para las columnas no es necesario pues estas se separan en el \texttt{split()}.

\begin{minted}{python}
filas, columnas = map(int, input().split())

tablero = [[int(x) for x in input().split()] for _ in range(filas)]
\end{minted}

\subsection{Imprimir elementos separados por espacio dentro de una lista}

El elemento de salida de este bloque de código corresponde a la concatenación de los strings separados por un espacio.

\begin{minted}{python}
lista = [string1, string2, string3]
stringConcatenado = ' '.join(lista)
\end{minted}

\subsection{Formato para floats con el juez en línea}

Se debe utilizar el siguiente formato para tener un redondeo con $n$ decimales (se sustituye el $n$ por la cantidad requerida).

\begin{minted}{python}
print(f"{round(numero), n}:.nf")
\end{minted}

\section{Algoritmos generales}

\subsection{Suma máxima de un subarreglo (\textit{Algoritmo de Kadane})}

Se tiene un arreglo numérico, al cual se requiere obtener la suma máxima de un subarreglo (datos contiguos).
Este algoritmo posee una complejidad temporal $O(n)$.

\begin{minted}{python}
def algoritmo_kadane(arr):
    # Inicializar sumas con el primer elemento
    actual = mejor = arr[0]

    for num in arr[1:]:
        actual = max(num, actual + num)
        mejor = max(mejor, actual)

    return mejor
\end{minted}

\subsection{Encontrar máximo o mínimo de un arreglo de datos}

Recuerde que existen las funciones integradas \texttt{max()} y \texttt{min()}.

Si se requiere encontrar el índice del número mayor puede utilizar \mintinline{python}{.index()} en listas o de la siguiente forma:

\begin{minted}{python}
indice_max = None
indice_min = None

for i in range(len(l)):
    # Encontrar el maximo
    if indice_max == None or indice_max < l[i]:
        indice_max = i

    # Encontrar el minimo
    if indice_min == None:
        indice_min = i
    elif indice_min > l[i]:
        indice_min = i
\end{minted}

\subsection{Sumas de prefijos de un arreglo}

La suma de prefijos de un arreglo consiste en una estructura de datos que guarda las sumas acumulativas del arreglo hasta cierto índice.

\subsubsection{Lineal}

En el caso lineal, se guardan las sumas acumulativas en un arreglo del mismo tamaño del original.

\begin{minted}{python}
# Lista con la suma acumulativa hasta cierto numero
prefijos = [0 for _ in range(len(numeros))]

prefijos[0] = numeros[0]

for i in range(1, len(numeros)):
    prefijos[i] = prefijos[i-1] + numeros[i]

\end{minted}

\subsubsection{Bidimensional}

En el caso del bidimensional, se guardan las sumas en un arreglo con la misma cantidad de entradas.
Se restan las prefijos, como se muestra en la línea \texttt{26} en adelante, de forma que se ajusta la suma dependiendo de la casilla en la que se encuentra para una consulta entre dos entradas distintas de la inicial del arreglo.

\begin{minted}{python}
filas = len(matriz)
columnas = len(matriz[0])

# Crear matriz de 0 del tamano filas x columnas (prefijos)
p = [[0 for _ in range(columnas)] for _ in range(filas)]

# Primer elemento de la suma
p[0][0] = matriz[0][0]

# Llenar primera fila
for i in range(1, filas):
    p[i][0] = p[i-1][0] + matriz[i][0]

# Llenar primera columna
for j in range(1, columnas):
    p[0][j] = p[0][j-1] + matriz[0][j]

# Llenar el resto de la matriz p
for i in range(1, filas):
    for j in range(1, columnas):
        p[i][j] = p[i-1][j] + p[i][j-1] - p[i-1][j-1] + matriz[i][j]

def sumaRango(f1, c1, f2, c2):
    suma = p[f2][c2]

    # Restar prefijos para ajustar suma
    if f1 > 0:
        suma -= p[f1-1][c2]
    if c1 > 0:
        suma -= p[f2][c1-1]
    if f1 > 0 and c1 > 0:
        suma += p[f1-1][c1-1]

    return suma
\end{minted}

\subsection{Búsqueda binaria}

El algoritmo de búsqueda binaria sirve para hallar un número objetivo dentro de un arreglo de datos.

\subsubsection{Iterativa}

Implementación con un \texttt{while} loop.

\begin{minted}{python}
def busqueda_binaria_iterativa(arreglo, objetivo):
    izquierdo, derecho = 0, len(arreglo) - 1

    while izquierdo <= derecho:
        medio = (izquierdo + derecho) // 2

        if arreglo[medio] == objetivo:
            return medio
        elif arreglo[medio] < objetivo:
            izquierdo = medio + 1
        else:
            derecho = medio - 1
    return -1 # Si no se encontro
\end{minted}

\subsubsection{Recursiva}

Implementación con recursión y definición de los casos base para la terminación del proceso recursivo.

\begin{minted}{python}
def busqueda_binaria_recursiva(arreglo, objetivo, izquierdo, derecho):
    # Si no se encontro (caso base)
    if izquierdo > derecho:
        return -1
    medio = (izquierdo + derecho) // 2
    if arreglo[medio] == objetivo:
        return medio
    elif arreglo[medio] < objetivo:
        return busqueda_binaria_recursiva(arreglo, objetivo, medio + 1, derecho)
    else:
        return busqueda_binaria_recursiva(arreglo, objetivo, izquierdo, medio - 1)
\end{minted}

\subsection{Ordenamiento de listas}

En Python, el método \mintinline{python}{.sort()} permite ordenar listas con base en un criterio de ordenamiento utilizando el parámetro \mintinline{python}{key}, que acepta una función.

\subsubsection{Ordenamiento de números}

\begin{enumerate}
  \item \textbf{Ordenar por valor absoluto:}

\begin{minted}{python}
numeros = [3, -7, 1, -2]
numeros.sort(key=lambda x: abs(x))
print(numeros)  # [1, -2, 3, -7]
\end{minted}

  \item \textbf{Ordenar por cantidad de dígitos:}

\begin{minted}{python}
numeros = [100, 5, 23, 9, 1000]
numeros.sort(key=lambda x: len(str(x)))
print(numeros)  # [5, 9, 23, 100, 1000]
\end{minted}

\end{enumerate}

\subsubsection{Ordenamiento con strings}

\begin{enumerate}

  \item \textbf{Por longitud:}

\begin{minted}{python}
palabras = ['gato', 'elefante', 'pez']
palabras.sort(key=lambda x: len(x))
print(palabras)  # ['pez', 'gato', 'elefante']
\end{minted}

  \item \textbf{Por última letra:}

\begin{minted}{python}
palabras = ['gato', 'elefante', 'pez']
palabras.sort(key=lambda x: x[-1])
print(palabras)  # ['elefante', 'gato', 'pez']
\end{minted}

\end{enumerate}

\subsubsection{Ordenamiento con tuplas}

\begin{enumerate}

  \item \textbf{Por segundo valor:}

\begin{minted}{python}
pares = [(1, 3), (2, 1), (5, 0)]
pares.sort(key=lambda x: x[1])
print(pares)  # [(5, 0), (2, 1), (1, 3)]
\end{minted}

  \item \textbf{Por suma de elementos:}

\begin{minted}{python}
puntos = [(1, 2), (3, 1), (0, 0)]
puntos.sort(key=lambda x: x[0] + x[1])
print(puntos)
\end{minted}

  \item \textbf{Por múltiples criterios: nombre y edad:} Se define el primer criterio que sea \mintinline{python}{x[0]} y el segundo \mintinline{python}{x[1]}.

\begin{minted}{python}
personas = [('Ana', 25), ('Luis', 20), ('Ana', 19)]
personas.sort(key=lambda x: (x[0], x[1]))
print(personas)
\end{minted}

  \item \textbf{Por edad descendente, luego nombre:}

\begin{minted}{python}
personas = [('Ana', 25), ('Luis', 25), ('Bea', 30)]
personas.sort(key=lambda x: (-x[1], x[0]))
print(personas)
\end{minted}

\end{enumerate}

\section{Algoritmos sobre números}

\subsection{Algoritmo de Euclides}

Sirve para el cálculo del máximo común divisor (\texttt{mcd}) de un número.

\begin{equation*}
\text{mcd}(a,b) =
     \begin{cases}
       \text{a} & \quad b = 0 \\
       \text{mcd}(b, a \text{ mód }b) &\quad b \neq 0 \\
     \end{cases}
\end{equation*}

\begin{minted}{python}
def euclides(a, b):
    while b != 0:
        a, b = b, a % b

    return a

def mcd(a,b):
    if b == 0:
        return a
    return mcd(b,a%b)
\end{minted}

Recuerde la propiedad que relaciona el mínimo común múltiplo y el máximo común divisor de dos números $a$ y $b$:

\begin{equation*}
    \text{mcm}(a,b) = \frac{a \cdot b}{\text{mcd}(a,b)}
\end{equation*}

\subsection{Exponenciación binaria}

Para calcular eficientemente una potencia se utiliza este algoritmo pues se ahorra casi la mitad de multiplicaciones en cada paso recursivo.

\begin{minted}{python}
def potencia(a, b):
	if b == 0:
		return 1

	s = potencia(a, b//2)

	if b % 2 == 0:
		return s*s
	else:
		return s * s * a
\end{minted}

\subsection{Test de primalidad \texorpdfstring{$O(\sqrt{n})$}{O(√n)}}

Para encontrar si un número es primo o no.
Se repite el proceso para todo $x^2 < n$; en cada repetición del \texttt{while} loop, se incrementa el valor de x.

\begin{minted}{python}
n = 25 # Numero de ejemplo
x = 2
esPrimo = True

while x*x <= n:
    if n%x == 0:
        esPrimo = False
        break
    x+=1

print(esPrimo)
\end{minted}

\subsection{Cálculo de divisores}

Se comprueba la divisibilidad de cada número en el rango de $\left[ 2, n \right]$, por medio de un \texttt{for} loop.
Cuando se encuentra que es divisible, se divide $n$ respecto a $i$.

\begin{minted}{python}
from math import sqrt

n = 16 # Numero de prueba
raiz = int(sqrt(n))

# Lista de divisores
divisores = [1]

if n != 1:
    divisores.append(n)

for i in range(2, raiz+1):
    if n % i == 0:
        divisores.append(i)

        # Agregar el complemento del divisor si es distinto de sí mismo
        if n // i != i:
            divisores.append(n//i)

divisores.sort() # Ordenamiento creciente
print(divisores)
\end{minted}

\subsection{Criba de Eratóstenes}

\subsubsection{Imprimir lista de primos}

Imprime una lista que muestra 0 para los números primos y su máximo divisor primo para los números compuestos.

\begin{minted}{python}
def criba_lista(n):
    lista_criba = list()

    # Creacion de la lista inicializada en 0 para n elementos
    for _ in range(2, n+1):
        lista_criba.append(0)

    # Implementacion del algoritmo
    for i in range(2, n+1):
        if lista_criba[i-2] != 0:
            continue

        for j in range(2*i, n+1, i):
            lista_criba[j-2] = i

    print(lista_criba)
\end{minted}

\subsubsection{Comprobación de primalidad de número específico}

Imprime si el número $n$ ingresado como parámetro es primo o no.
En caso de que no lo sea, imprime el menor primo que lo divide.

\begin{minted}{python}
def criba_verificar_primo(n):

    es_primo = [True for _ in range(n+1)]

    menor_primo = [-1 for _ in range(n+1)]

    es_primo[0] = es_primo[1] = False

    primos = []

    for i in range(2, n+1):
        if es_primo[i]:
            primos.append(i)

            for j in range(i*i, n+1, i):
                es_primo[j] = False

                if menor_primo[j] == -1:
                    menor_primo[j] = i


    print(es_primo[n], menor_primo[n])
\end{minted}

\subsection{Cálculo de la distancia entre puntos}

Suma de la distancia entre una serie de puntos que forman una figura cerrada.

\begin{minted}{python}
from math import sqrt

def dist(p1, p2):
  # Coordenadas
  x1, y1 = p1
  x2, y2 = p2

  return sqrt((x2-x1)**2 + (y2-y1)**2)
\end{minted}

\section{Programación Dinámica}

Consiste en el principio de programación donde se divide un problema en subproblemas para resolverlos y utilizar las soluciones previas para construir la general.

\subsection{Memoización con \texttt{@lru\_cache}}

Utiliza el decorador \texttt{@lru\_cache} para guardar el resultado obtenido de cada paso recursivo dentro del programa.

\begin{minted}{python}
from functools import lru_cache

@lru_cache(maxsize=None)
def fibonacci(n):
    if n == 0:
        return 0
    if n == 1:
        return 1
    print(f"Calculo de {n}")
    return fibonacci(n-1) + fibonacci(n-2)
\end{minted}

% ==================

\section{Algoritmos sobre grafos}

Los tipos de recubrimiento en grafos estudiados para la olimpiada consisten en los siguientes:

\begin{enumerate}
    \item Recubrimiento por profundidad: Explora un camino hasta el final (un vértice sin vecinos no visitados) antes de retroceder y explorar otros caminos. Utiliza \textbf{pilas}.
    \item Recubrimiento por anchura: Visita los nodos en niveles y utiliza \textbf{colas}.
\end{enumerate}

Entre los conceptos de interés para el estudio de grafos, se tienen a continuación:
\begin{description}
    \item[Matriz de adyacencia:] Consiste en una matriz de tamaño $n \times n$ para un grafo con $n$ nodos, tal que cada entrada contiene un $1$ en caso de que haya una arista entre el par de nodos de los índices, o un $0$ si no hay conexión.
    \item[Lista de Adyacencia:] Cada índice $i$ contiene una lista con las aristas del nodo $i$.
\end{description}

\subsection{Recubrimiento por profundidad (DFS)}

\subsubsection{Por matriz de adyacencia}

\begin{minted}{python}
def dfs(grafo):
    pila = [0]  # Empieza desde el nodo 0 por ejemplo
    visitados = [0] * len(grafo)  # Marca si el nodo fue visitado y el orden en que fue visitado
    cont = 1  # Contador para registrar el orden de visita

    while pila:
        top = pila.pop()  # Extrae el nodo del top de la pila
        if not visitados[top]:
            visitados[top] = cont  # Marca el nodo como visitado con el contador
            cont += 1
            # Recorre los vecinos (desde el nodo más alto al más bajo)
            for i in reversed(range(len(grafo[top]))):
                if grafo[top][i] == 1 and not visitados[i]:
                    pila.append(i)  # Agrega vecinos no visitados a la pila
    print(visitados)
\end{minted}

\subsubsection{Por lista de adyacencia}

\begin{minted}{python}
def dfs(grafo, nodo, visitado=None):
    if visitado is None:
        visitado = set()  # Conjunto para registrar nodos ya visitados (evita repeticiones)
    visitado.add(nodo)
    print(nodo, end=' ')  # Procesa el nodo actual

    # Recorre todos los vecinos del nodo actual
    for vecino in grafo[nodo]:
        if vecino not in visitado:
            dfs(grafo, vecino, visitado)  # Llamada recursiva sobre el vecino no visitado
\end{minted}

\subsection{Recubrimiento por anchura (BFS)}

\subsubsection{Por matriz de adyacencia}

Como nota, en esta implementación, por la naturaleza de la listas, se insertan los elementos por el lado izquierdo de la lista (al inicio) y se eliminan por el derecho.

\begin{minted}{python}
from collections import deque

def bfs(grafo):
    cola = deque()
    cola.append(0)
    visitados = [0] * len(grafo)
    cont = 1

    while cola:
        front = cola.popleft()
        if not visitados[front]:
            visitados[front] = cont
            cont += 1
            for i in range(len(grafo[front])):
                if grafo[front][i] == 1 and not visitados[i]:
                    cola.append(i)
    print(visitados)
\end{minted}

\subsubsection{Por lista de adyacencia}

Se utiliza la estructura de datos importada del módulo \texttt{Collections} denominada \texttt{deque}, que corresponde a la implementación de una cola.
De esta forma, se sigue la idea original del orden de inserción y eliminación de elementos.

\begin{minted}{python}
from collections import deque

def bfs(grafo, nodo_inicio):
    visitado = set()
    cola = deque([nodo_inicio]) # Inicializar cola
    while cola:
        nodo = cola.popleft()
        if nodo not in visitado:
            print(nodo, end=' ')
            visitado.add(nodo)
            # Agrega vecinos no visitados al final de la cola
            cola.extend([vec for vec in grafo[nodo] if vec not in visitado])
\end{minted}

\subsection{Algoritmos sobre grafos}

\subsubsection{Detección de ciclos}

\begin{minted}{python}
def tiene_ciclo(grafo):
    visitado = set()

    def dfs(nodo, padre):
        visitado.add(nodo)
        for vecino in grafo[nodo]:
            if vecino not in visitado:
                if dfs(vecino, nodo):
                    return True
            elif vecino != padre:
                return True
        return False

    for nodo in grafo:
        if nodo not in visitado:
            if dfs(nodo, None):
                return True
    return False
\end{minted}

\subsubsection{Verificar existencia de camino entre una serie de nodos}

A partir de una secuencia que se recibe en forma de lista o de string, que contiene a cada uno de los nodos, en orden, que se requiere determinar si existe un camino para conectarlos.
Por medio de recursión, se retorna el \texttt{True} o \texttt{False} al terminar el proceso.

\begin{minted}{python}
def esCamino(sec, grafo):
    # Para matriz de adyacencia
    if len(sec) < 2:
        return True
    if grafo[sec[0]][sec[1]] == 0:
        return False
    return esCamino(sec[1:], grafo)
\end{minted}

\subsubsection{Comprobar si un grafo es conexo}

Un grafo conexo corresponde a un grafo tal que no existen que estén aislados del resto del grafo; es decir, todos los nodos están conectados por, al menos, un camino.

\begin{minted}{python}
def componentes_conexas(grafo):
    visitado = set()
    componentes = []

    def dfs(nodo, componente):
        visitado.add(nodo)
        componente.append(nodo)
        for vecino in grafo[nodo]:
            if vecino not in visitado:
                dfs(vecino, componente)

    for nodo in grafo:
        if nodo not in visitado:
            componente = []
            dfs(nodo, componente)
            componentes.append(componente)

    return componentes
\end{minted}

\subsubsection{Orden topológico de nodos}

\begin{minted}{python}
grafo = {
    "A": ["C"],
    "B": ["C", "D"],
    "C": ["E"],
    "D": ["F"],
    "E": ["H", "F"],
    "F": ["G"],
    "G": [],
    "H": [],
}

def orden_topologico(grafo):
    visitado = set()
    pila = []

    def dfs(nodo):
        visitado.add(nodo)
        for vecino in grafo.get(nodo, []):
            if vecino not in visitado:
                dfs(vecino)
        pila.append(nodo)

    for nodo in grafo:
        if nodo not in visitado:
            dfs(nodo)

    return pila[::-1]  # Invierte la pila para obtener el orden topológico correcto


print(orden_topologico(grafo))
\end{minted}

\subsubsection{Puntos de articulación y puentes}

\begin{minted}{python}
def articulaciones_y_puentes(grafo):
    tiempo = 0
    visitado = set()
    disc = {}
    low = {}
    padre = {}
    articulaciones = set()
    puentes = []

    def dfs(u):
        nonlocal tiempo
        visitado.add(u)
        disc[u] = low[u] = tiempo
        tiempo += 1
        hijos = 0

        for v in grafo[u]:
            if v not in visitado:
                padre[v] = u
                hijos += 1
                dfs(v)
                # Si `v` o un descendiente alcanza un nodo descubierto antes que `u`, entonces `u`
                # también puede alcanzarlo y se actualiza con este valor
                low[u] = min(low[u], low[v])

                # Condición para punto de articulación
                if padre.get(u) is None and hijos > 1:
                    articulaciones.add(u)

                # Si no existe un camino desde `v` o sus descendientes hacia un nodo descubierto antes que `u`
                # entonces el único camino hacia los anteriores, sería a través de `u`
                if padre.get(u) is not None and low[v] >= disc[u]:
                    articulaciones.add(u)

                # Condición para puente
                # Si no se alcanza ningún nodo descubierto antes que `u`
                if low[v] > disc[u]:
                    puentes.append((u, v))
            elif v != padre.get(u):
                # Caso de arista de retroceso
                #
                low[u] = min(low[u], disc[v])

    for u in grafo:
        if u not in visitado:
            dfs(u)

    return articulaciones, puentes
\end{minted}

\end{document}
