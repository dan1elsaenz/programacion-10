\section*{Parte 2: Programación funcional}

Para esta parte, debe entregar un archivo \texttt{parte2.txt}, donde enumere las respuestas correctas como:
\begin{minted}{text}
1. X
2. X
3. X
4. X
5. X
6. <Explicación>
\end{minted}

Cada pregunta de selección única tiene un valor de $6\%$ y la pregunta de desarrollo tiene un valor de $10\%$.

\subsubsection*{Pregunta 1}

¿Cuál comprensión de lista \textbf{equivale} a quedarse con las palabras que empiezan con \texttt{a} y ponerlas en mayúsculas?

\begin{minted}{python}
palabras = ["ana", "pedro", "alberto", "lucía"]
\end{minted}

\begin{enumerate}[label=\alph*)]
  \item \mintinline{python}{[p for p in palabras if p.startswith("a")]}
  \item \mintinline{python}{[p.upper() for p in palabras if p.startswith("a")]}
  \item \mintinline{python}{[p.upper() if p.startswith("a") for p in palabras]}
  \item \mintinline{python}{[p for p in palabras.upper() if p.startswith("a")]}
\end{enumerate}

\subsubsection*{Pregunta 2}

¿Cuál expresión genera la suma de los cuadrados de los números positivos en la lista?

\begin{minted}{python}
nums = [-2, -1, 0, 1, 2, 3]
\end{minted}

\begin{enumerate}[label=\alph*)]
  \item \mintinline{python}{sum([x**2 for x in nums if x > 0])}
  \item \mintinline{python}{sum(x**2 for x in nums if x > 0)}
  \item \mintinline{python}{sum(map(lambda x: x**2, filter(lambda x: x > 0, nums)))}
  \item Todas las anteriores producen el mismo resultado.
\end{enumerate}

\subsubsection*{Pregunta 3}

¿Cuál expresión ordena la lista de palabras según la última letra de cada una?
\begin{minted}{python}
palabras = ["python", "java", "c", "javascript"]
\end{minted}

\begin{enumerate}[label=\alph*)]
  \item \mintinline{python}{sorted(palabras, key=lambda s: s[0])}
  \item \mintinline{python}{sorted(palabras, key=lambda s: s[-1])}
  \item \mintinline{python}{sorted(palabras, key=len)}
  \item \mintinline{python}{sorted(palabras)}
\end{enumerate}

\subsubsection*{Pregunta 4}

¿Cuál de las siguientes expresiones con funciones de orden superior obtiene sólo los números negativos?
\begin{minted}{python}
nums = [-2, -1, 0, 1, 2]
\end{minted}

\begin{enumerate}[label=\alph*)]
  \item \mintinline{python}{list(filter(lambda x: x < 0, nums))}
  \item \mintinline{python}{list(filter(lambda x: x <= 0, nums))}
  \item \mintinline{python}{list(filter(lambda x: x > 0, nums))}
  \item \mintinline{python}{[x for x in nums if x < 0]}
\end{enumerate}

\subsubsection*{Pregunta 5}

¿Cuál expresión genera una lista con los dobles de los números dados?
\begin{minted}{python}
nums = [1, 2, 3]
\end{minted}

\begin{enumerate}[label=\alph*)]
  \item \mintinline{python}{map(lambda x: x*2, nums)}
  \item \mintinline{python}{list(map(lambda x: x*2, nums))}
  \item \mintinline{python}{[x*2 for x in nums]}
  \item Las opciones b) y c) son equivalentes.
\end{enumerate}

\subsubsection*{Pregunta 6}
Explique la diferencia entre una \textbf{comprensión de lista} y una \textbf{expresión generadora} en términos de evaluación y uso de memoria.
Indique un caso concreto en el que conviene usar cada una.
