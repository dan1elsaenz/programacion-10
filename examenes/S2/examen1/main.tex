\documentclass[letterpaper,12pt]{article}

\usepackage[spanish]{babel}
\usepackage[utf8]{inputenc}

% \usepackage[letterpaper,margin=2.5cm]{geometry}
\usepackage[letterpaper, top=2.0cm, bottom=2.5cm, left=2.5cm, right=2.5cm]{geometry}
\usepackage{parskip}
\usepackage{fancyhdr}

% Fonts
\usepackage{lmodern}
% \usepackage{newcent}
% \usepackage{mathptmx}
% \usepackage{times}
% \usepackage{tgbonum}
% \usepackage{cmbright}

\usepackage{lastpage}
\usepackage{graphicx}
\usepackage{amsthm}
\usepackage{caption}
\usepackage{multirow}
\usepackage{multicol}
\usepackage{titlesec}
\usepackage{enumitem}
\usepackage{longtable}
\usepackage{array}
\usepackage[dvipsnames]{xcolor}

\usepackage[colorlinks,pagebackref,pdfusetitle,urlcolor=NavyBlue,citecolor=darkblue,linkcolor=.]{hyperref}

\renewcommand{\thefootnote}{\fnsymbol{footnote}}

% Cambiar segundo nivel de enumerate con el formato #.#
\renewcommand{\theenumii}{\theenumi.\arabic{enumii}}
\renewcommand{\labelenumii}{\theenumii.}

\setlist[enumerate,1]{label=\arabic*.}
\setlist[enumerate,2]{label*=\arabic*}


\begin{document}

\AddToShipoutPicture{\BackgroundPic} % Agrega el fondo a todas las hojas

\thispagestyle{firstpage}

% Encabezado
\noindent
\begin{minipage}[t]{0.59\textwidth}
    \textbf{Colegio Científico de Puriscal} \\
    \textbf{Programación 10°} \\
    \textbf{Ciclo:} II-2025
\end{minipage}
\begin{minipage}[t]{0.4\textwidth}
    \raggedleft
    \textbf{Fecha:} 22/10/2025 \\
    \textbf{Profesor:} Daniel Sáenz Obando
\end{minipage}

\vspace{0.2cm}
\hrule
\vspace{0.1cm}

\begin{center}
	\Large \textbf{Examen Parcial I}
\end{center}

\vspace{0.1cm}
\hrule
\vspace{0.2cm}

\subsection*{Indicaciones generales}

\begin{itemize}
    \item El examen tiene una duración total de 2 horas y 15 minutos. Subir al Google Classroom las soluciones en el espacio denominado \texttt{Examen Parcial I}.
    \item Debe entregar los siguientes archivos: \texttt{parte1.py} y \texttt{parte2.txt}.
    \begin{itemize}
        \item Subir cada uno de los archivos al espacio, no como una carpeta comprimida.
    \end{itemize}
    \item Esta es una prueba de carácter individual y está prohibido el acceso a información en internet. En caso de detectar uso de inteligencia artificial, se va a anular la nota para dicha persona.
\end{itemize}

\hrule

\section*{Parte 1: Carrito de compras de una librería en línea}

\subsection*{{\small \faHistory}\; Contexto}

Amazon inició en 1994 como una \textit{librería en línea} enfocada en la venta de libros físicos mediante un carrito de compras.
En este examen, debe implementar una versión simplificada de ese \textbf{carrito de compras} en Python para \emph{libros físicos} y \emph{ebooks}, por medio del paradigma de programación orientada a objetos.

Debe entregar un archivo llamado \texttt{parte1.py} como solución de esta parte.

\subsection*{{\small \faCubes}\; Requerimientos por clase}

En este apartado, se describen cada uno de los requerimientos técnicos a implementar en el diseño.

\subsubsection*{Clase \texttt{Producto}}

\begin{itemize}
  \item \textbf{Atributos:}
    \begin{itemize}
      \item \texttt{nombre} (\texttt{str}).
      \item \texttt{precio} (numérico) como \textit{atributo privado}.
    \end{itemize}

  \item \textbf{Encapsulamiento de \texttt{precio}:} implementar \mintinline{python}{@property} y \mintinline{python}{@precio.setter}.
    El setter debe impedir precios negativos; si se asigna un valor negativo, el atributo de \mintinline{python}{precio} debe quedar en \mintinline{python}{0}.

  \item \textbf{Representación:} definir \mintinline{python}{__str__} para devolver:
\begin{minted}{text}
Producto: <nombre>, Precio: <precio>
\end{minted}

  \item \textbf{Sobrecarga de operadores:}
    \begin{itemize}
      \item \mintinline{python}{__add__}: al sumar dos productos, debe devolver la suma de sus precios.
    \end{itemize}
\end{itemize}

\subsubsection*{Clase \texttt{LibroFisico}}

\begin{itemize}
  \item \textbf{Herencia:} debe heredar de \texttt{Producto}.
  \item \textbf{Atributo adicional:} \texttt{masa} (en kilogramos).
  \item \textbf{Representación:} redefinir \mintinline{python}{__str__} para devolver exactamente:
\begin{minted}{text}
Libro Físico: <nombre>, Precio: <precio>, Masa: <masa> kg
\end{minted}
\end{itemize}

\subsubsection*{Clase \texttt{Ebook}}

\begin{itemize}
  \item \textbf{Herencia:} debe heredar de \texttt{Producto}.
  \item \textbf{Atributo adicional:} \texttt{formato} (por ejemplo: \mintinline{python}{"PDF"}, \mintinline{python}{"EPUB"}).
  \item \textbf{Representación:} redefinir \mintinline{python}{__str__} para devolver exactamente:
\begin{minted}{text}
Ebook: <nombre>, Precio: <precio>, Formato: <formato>
\end{minted}
\end{itemize}

\subsubsection*{Clase \texttt{Carrito}}

\begin{itemize}
  \item \textbf{Atributo:} \texttt{productos}, una lista que almacena instancias de \mintinline{python}{LibroFisico} y \mintinline{python}{Ebook} (heredadas de \mintinline{python}{Producto}).

  \item \textbf{Métodos:}
    \begin{itemize}
      \item \mintinline{python}{agregar_producto(producto)}: Agrega un objeto de tipo \texttt{Producto} a la lista interna (acepta \texttt{Producto}, \texttt{LibroFisico} o \texttt{Ebook}).
      \item \mintinline{python}{vaciar_carrito()}: Elimina todos los productos del carrito.
      \item \mintinline{python}{calcular_total()}: Retorna la suma de todos los \mintinline{python}{precio} de los productos contenidos, por medio de un recorrido sobre los elementos de la lista.\footnote{\textit{Tip}: Anteriormente se implementó la sobrecarga de operadores para la suma, utilícela.}
      \item \mintinline{python}{mostrar_carrito()}: Recorre la lista e imprime cada elemento usando \mintinline{python}{print(obj)}.\footnote{\textit{Tip}: Se implementó la sobrecarga de operadores para \mintinline{python}{__str__}.}
    \end{itemize}

  \item \textbf{Sobrecarga de operadores:}
    \begin{itemize}
      \item \mintinline{python}{__len__}: Permite que \mintinline{python}{len(carrito)} devuelva la cantidad de productos almacenados. Es decir, al pasarle una instancia de la clase \mintinline{python}{Carrito} a la función \mintinline{python}{len()}, devuelve el largo de la lista \mintinline{python}{productos}.
    \end{itemize}

\end{itemize}

\subsection*{{\small \faPlay\;} Prueba en \texttt{main}}

Dentro de un bloque \mintinline{python}{if __name__ == "__main__":} se debe:\footnote{Observe que no se están realizando \mintinline{python}{input()}.}
\begin{enumerate}
  \item Crear al menos dos instancias de \texttt{LibroFisico} y una instancia de \texttt{Ebook}.
  \item Crear un objeto \texttt{Carrito} y agregar los productos creados.
  \item Mostrar el contenido del carrito con \mintinline{python}{mostrar_carrito()}.
  \item Imprimir el total con \mintinline{python}{calcular_total()}.
  \item Imprimir la cantidad de productos con \mintinline{python}{len(carrito)}.
  \item Vaciar el carrito y verificar que quede vacío.
\end{enumerate}

\section*{{\small \faClipboardCheck\;} Rúbrica de evaluación}

\begin{table}[H]
    \centering
    \resizebox{0.6\textwidth}{!}{
    \begin{tabular}{p{11cm}r}
        \toprule
        \textbf{Criterios} & \textbf{Puntos} \\
        \midrule
        \multicolumn{2}{l}{\textbf{Clase \texttt{Producto} (15 pts)}}\\
        Definición de atributos (\mintinline{python}{nombre}, \mintinline{python}{precio} privado) & 4 \\
        Encapsulamiento \mintinline{python}{precio} con \mintinline{python}{@property} y setter (bloquea negativos) & 6 \\
        Implementación de \mintinline{python}{__str__} con formato requerido & 3 \\
        Implementación de \mintinline{python}{__add__} para suma & 2 \\
        \midrule
        \multicolumn{2}{l}{\textbf{Clases \texttt{LibroFisico} y \texttt{Ebook} (15 pts)}}\\
        \texttt{LibroFisico}: atributo \mintinline{python}{masa} y \mintinline{python}{__str__} actualizado & 7 \\
        \texttt{Ebook}: atributo \mintinline{python}{formato} y \mintinline{python}{__str__} actualizado & 8 \\
        \midrule
        \multicolumn{2}{l}{\textbf{Clase \texttt{Carrito} (20 pts)}}\\
        Atributo \mintinline{python}{productos} & 2 \\
        \mintinline{python}{agregar_producto} agrega instancias válidas & 4 \\
        \mintinline{python}{vaciar_carrito} vacía completamente & 3 \\
        \mintinline{python}{calcular_total} suma correctamente los precios & 6 \\
        \mintinline{python}{mostrar_carrito} imprime usando polimorfismo & 3 \\
        \mintinline{python}{__len__} retorna la cantidad de productos & 2 \\
        \midrule
        \multicolumn{2}{l}{\textbf{Prueba en \texttt{main} (10 pts)}}\\
        Crea 2 \texttt{LibroFisico} y un \texttt{Ebook} & 3 \\
        Agrega al carrito y muestra con \mintinline{python}{mostrar_carrito} & 2 \\
        Imprime total (\mintinline{python}{calcular_total}) & 1 \\
        Imprime cantidad (\mintinline{python}{len(carrito)}) & 1 \\
        Vacía y verifica carrito vacío & 3 \\
        \midrule
        \textbf{Total} & \textbf{60} \\
        \bottomrule
    \end{tabular}
    }
    \caption{Rúbrica de evaluación del Ejercicio 1.}
    \label{tab1}
\end{table}

\noindent\textbf{\faInfoCircle\; Nota:} Se acepta cualquier diseño interno mientras se cumplan exactamente las salidas y restricciones indicadas por clase.

\section*{Parte 2: Programación funcional}

Para esta parte, debe entregar un archivo \texttt{parte2.txt}, donde enumere las respuestas correctas como:
\begin{minted}{text}
1. X
2. X
3. X
4. X
5. X
6. <Explicación>
\end{minted}

Cada pregunta de selección única tiene un valor de $6\%$ y la pregunta de desarrollo tiene un valor de $10\%$.

\subsubsection*{Pregunta 1}

¿Cuál comprensión de lista \textbf{equivale} a quedarse con las palabras que empiezan con \texttt{a} y ponerlas en mayúsculas?

\begin{minted}{python}
palabras = ["ana", "pedro", "alberto", "lucía"]
\end{minted}

\begin{enumerate}[label=\alph*)]
  \item \mintinline{python}{[p for p in palabras if p.startswith("a")]}
  \item \mintinline{python}{[p.upper() for p in palabras if p.startswith("a")]}
  \item \mintinline{python}{[p.upper() if p.startswith("a") for p in palabras]}
  \item \mintinline{python}{[p for p in palabras.upper() if p.startswith("a")]}
\end{enumerate}

\subsubsection*{Pregunta 2}

¿Cuál expresión genera la suma de los cuadrados de los números positivos en la lista?

\begin{minted}{python}
nums = [-2, -1, 0, 1, 2, 3]
\end{minted}

\begin{enumerate}[label=\alph*)]
  \item \mintinline{python}{sum([x**2 for x in nums if x > 0])}
  \item \mintinline{python}{sum(x**2 for x in nums if x > 0)}
  \item \mintinline{python}{sum(map(lambda x: x**2, filter(lambda x: x > 0, nums)))}
  \item Todas las anteriores producen el mismo resultado.
\end{enumerate}

\subsubsection*{Pregunta 3}

¿Cuál expresión ordena la lista de palabras según la última letra de cada una?
\begin{minted}{python}
palabras = ["python", "java", "c", "javascript"]
\end{minted}

\begin{enumerate}[label=\alph*)]
  \item \mintinline{python}{sorted(palabras, key=lambda s: s[0])}
  \item \mintinline{python}{sorted(palabras, key=lambda s: s[-1])}
  \item \mintinline{python}{sorted(palabras, key=len)}
  \item \mintinline{python}{sorted(palabras)}
\end{enumerate}

\subsubsection*{Pregunta 4}

¿Cuál de las siguientes expresiones con funciones de orden superior obtiene sólo los números negativos?
\begin{minted}{python}
nums = [-2, -1, 0, 1, 2]
\end{minted}

\begin{enumerate}[label=\alph*)]
  \item \mintinline{python}{list(filter(lambda x: x < 0, nums))}
  \item \mintinline{python}{list(filter(lambda x: x <= 0, nums))}
  \item \mintinline{python}{list(filter(lambda x: x > 0, nums))}
  \item \mintinline{python}{[x for x in nums if x < 0]}
\end{enumerate}

\subsubsection*{Pregunta 5}

¿Cuál expresión genera una lista con los dobles de los números dados?
\begin{minted}{python}
nums = [1, 2, 3]
\end{minted}

\begin{enumerate}[label=\alph*)]
  \item \mintinline{python}{map(lambda x: x*2, nums)}
  \item \mintinline{python}{list(map(lambda x: x*2, nums))}
  \item \mintinline{python}{[x*2 for x in nums]}
  \item Las opciones b) y c) son equivalentes.
\end{enumerate}

\subsubsection*{Pregunta 6}
Explique la diferencia entre una \textbf{comprensión de lista} y una \textbf{expresión generadora} en términos de evaluación y uso de memoria.
Indique un caso concreto en el que conviene usar cada una.


\end{document}
