% Para escribir tildes y eñes
\usepackage[utf8]{inputenc}
\usepackage[T1]{fontenc}

% Tamaño del área de escritura de la página
\usepackage[margin=2.5cm]{geometry}

% Para que los títulos de figuras, tablas y otros estén en español
\usepackage[spanish]{babel}

% Letra
\usepackage{libertinus}
\usepackage{libertinust1math}

\usepackage{FiraMono}
\renewcommand{\ttdefault}{fvm}
\DeclareTextFontCommand{\texttt}{\footnotesize\ttfamily}

% Letra para el código
\usepackage{minted}
\usemintedstyle{emacs}

% Configuración para minted con fuente FiraMono
\setminted{
  fontfamily=fvm, % FiraMono
  linenos,
  frame=lines,
  framesep=2mm,
  baselinestretch=1.2,
  fontsize=\footnotesize,
  breaklines=true,
  tabsize=4,
  numbersep=5pt,
  xleftmargin=1.5em,
  xrightmargin=1.5em
}
\setmintedinline{breaklines}

% Íconos
\usepackage{fontawesome5}

% Los paquetes ams son desarrollados por la American Mathematical Society y mejoran la escritura de fórmulas y símbolos matemáticos.
\usepackage{amsmath}
\usepackage{amsfonts}
\usepackage{amssymb}

% Para insertar gráficas
\usepackage{graphicx}

% Enums
\usepackage{enumitem}

% Para insertar hipervínculos y marcadores
\usepackage[colorlinks=true,urlcolor=blue,linkcolor=black,citecolor=green]{hyperref}

% Para ubicar las tablas y figuras justo después del texto
\usepackage{float}

% Para hacer tablas más estilizadas
\usepackage{booktabs}

% Para manejar los encabezados y pies de página
\usepackage{fancyhdr}
\usepackage{lastpage}

\pagestyle{fancy}
\fancyhf{}
\setlength{\headheight}{14.49998pt}
\fancyhead[L]{Examen Parcial I}
\fancyhead[R]{Programación 10°}
\fancyfoot[C]{\thepage\ de \pageref{LastPage}}


% Encabezado para la primera página
\fancypagestyle{firstpage}{
    \fancyhf{}
    \cfoot{\thepage\ de \pageref{LastPage}} % Número de página visible
    \renewcommand{\headrulewidth}{0pt} % Sin línea
}

% Cambiar nombre a tablas
\addto\captionsspanish{\renewcommand{\tablename}{Tabla}}

\renewcommand{\tablename}{Tabla}

% Imagen de fondo
\usepackage{eso-pic}
\usepackage{transparent}

\newcommand\BackgroundPic{
  \put(0,0){
    \parbox[b][\paperheight]{\paperwidth}{%
      \vfill
      \centering
      \transparent{0.05}  % Cambia el valor según lo que desees
      \includegraphics[width=0.5\paperwidth,keepaspectratio]{images/logo.jpg}
      \vfill
    }
  }
}
