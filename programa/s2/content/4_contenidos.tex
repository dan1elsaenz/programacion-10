\section{Contenidos del curso}

\begin{enumerate}

  \item \textbf{Programación orientada a objetos en Python}

  \begin{enumerate}
    \item \label{itm:1.1} Definición y uso de clases y objetos.
    \item \label{itm:1.2} Atributos y métodos de instancia.
    \item \label{itm:1.3} Encapsulamiento de datos.
    \item \label{itm:1.4} Herencia y reutilización de código.
    \item \label{itm:1.5} Polimorfismo y redefinición de métodos.
    \item \label{itm:1.6} Uso de decoradores: \texttt{@staticmethod}, \texttt{@classmethod}, \texttt{@property}.
    \item \label{itm:1.7} Sobrecarga de operadores en clases personalizadas.
  \end{enumerate}

  \item \textbf{Programación funcional en Python}

  \begin{enumerate}
    \item \label{itm:2.1} Concepto de funciones de orden superior y programación declarativa.
    \item \label{itm:2.2} Uso de funciones integradas como \texttt{map()}, \texttt{filter()} y \texttt{reduce()}.
    \item \label{itm:2.3} Expresiones lambda: definición y uso en operaciones funcionales.
    \item \label{itm:2.4} Comprensiones de listas y expresiones generadoras.
  \end{enumerate}

  \item \textbf{Árboles binarios}

  \begin{enumerate}
    \item \label{itm:3.1} Definición de nodos y estructura general de un árbol binario.
    \item \label{itm:3.2} Implementación de árboles binarios con clases.
    \item \label{itm:3.3} Recorridos: en orden, preorden, postorden (DFS).
    \item \label{itm:3.4} Cálculo de altura y verificación de balance.
    \item \label{itm:3.5} Introducción a árboles binarios de búsqueda (BST).
  \end{enumerate}

  \item \textbf{Grafos}

  \begin{enumerate}
    \item \label{itm:4.1} Representación de grafos: listas de adyacencia y matriz de adyacencia.
    \item \label{itm:4.2} Recorrido en profundidad (DFS) y recorrido en anchura (BFS).
    \item \label{itm:4.3} Detección de ciclos en grafos dirigidos y no dirigidos.
    \item \label{itm:4.4} Modelado de problemas con grafos (mapas, redes, dependencias).
  \end{enumerate}

  \item \textbf{Control de versiones con Git y GitHub}

  \begin{enumerate}
    \item \label{itm:5.1} Inicialización de repositorios locales.
    \item \label{itm:5.2} Comandos básicos: \texttt{git add}, \texttt{commit}, \texttt{push}, \texttt{pull}.
    \item \label{itm:5.3} Clonación y actualización de proyectos desde GitHub.
    \item \label{itm:5.4} Estructura de un repositorio con README y documentación básica.
    \item \label{itm:5.5} Colaboración en repositorios de GitHub.
    \item \label{itm:5.6} Manejo de conflictos.
  \end{enumerate}

  \item \textbf{Introducción a interfaces gráficas con Tkinter}

  \begin{enumerate}
    \item \label{itm:6.1} Creación de ventanas, botones, etiquetas y campos de entrada.
    \item \label{itm:6.2} Gestión de eventos (\textit{callbacks}) asociados a acciones del usuario.
    \item \label{itm:6.3} Organización del \textit{layout} y navegación entre vistas.
    \item \label{itm:6.4} Integración de la lógica del programa con la GUI.
  \end{enumerate}

  \item \textbf{Introducción a bases de datos con \texttt{sqlite3}}

  \begin{enumerate}
    \item \label{itm:7.1} Concepto de base de datos y sistema gestor.
    \item \label{itm:7.2} Creación y conexión a una base de datos \texttt{SQLite} desde Python.
    \item \label{itm:7.3} Ejecución de operaciones básicas: inserción, consulta, actualización y eliminación (CRUD).
  \end{enumerate}

\end{enumerate}

