\section{Contenidos del curso}

\begin{enumerate}

  \item \textbf{Fundamentos de la programación}

  \begin{enumerate}
    \item \label{itm:1.1} Concepto de programación y algoritmo.
    \item \label{itm:1.2} Lenguajes de programación y paradigmas.
    \item \label{itm:1.3} Ciclo de desarrollo de un programa: análisis, diseño, implementación y prueba.
    \item \label{itm:1.4} Construcción y ejecución de programas.
  \end{enumerate}

  \item \textbf{Variables, tipos de datos y operadores}

  \begin{enumerate}
    \item \label{itm:2.1} Variables, constantes y estado en memoria.
    \item \label{itm:2.2} Tipos de datos básicos.
    \item \label{itm:2.3} Asignación, conversión de tipos y entrada de datos.
    \item \label{itm:2.4} Operadores aritméticos, relacionales y lógicos.
    \item \label{itm:2.5} Precedencia y evaluación de expresiones.
  \end{enumerate}

  \item \textbf{Entrada, salida y validación de datos}

  \begin{enumerate}
    \item \label{itm:3.1} Entrada interactiva y salida estándar.
    \item \label{itm:3.2} Verificación de datos.
    \item \label{itm:3.3} Manejo básico de errores y excepciones.
  \end{enumerate}

  \item \textbf{Estructuras de control}

  \begin{enumerate}
    \item \label{itm:4.1} Secuencia y bloques de instrucciones.
    \item \label{itm:4.2} Condicionales: \texttt{if}, \texttt{elif}, \texttt{else}.
    \item \label{itm:4.3} Repetición: \texttt{for} y \texttt{while}.
    \item \label{itm:4.4} Patrones comunes de resolución con ciclos.
  \end{enumerate}

  \item \textbf{Funciones y modularidad}

  \begin{enumerate}
    \item \label{itm:5.1} Definición e invocación de funciones.
    \item \label{itm:5.2} Parámetros, retorno y alcance de variables.
    \item \label{itm:5.3} Descomposición de problemas y reutilización de código.
    \item \label{itm:5.4} Introducción a la recursividad.
  \end{enumerate}

  \item \textbf{Strings (cadenas de caracteres)}

  \begin{enumerate}
    \item \label{itm:6.1} Concepto y representación de cadenas.
    \item \label{itm:6.2} Operaciones básicas: concatenación, longitud y acceso.
    \item \label{itm:6.3} Métodos comunes de procesamiento de texto.
    \item \label{itm:6.4} Recorrido y análisis de cadenas.
  \end{enumerate}

  \item \textbf{Colecciones lineales de datos}

  \begin{enumerate}
    \item \label{itm:7.1} Listas: concepto, declaración e inicialización.
    \item \label{itm:7.2} Acceso, recorrido y modificación de elementos.
    \item \label{itm:7.3} Operaciones comunes: búsqueda, suma, promedio, mínimo y máximo.
    \item \label{itm:7.4} Introducción a diccionarios, tuplas y conjuntos.
  \end{enumerate}

  \item \textbf{Matrices (arreglos multidimensionales)}

  \begin{enumerate}
    \item \label{itm:8.1} Concepto y estructura.
    \item \label{itm:8.2} Declaración, inicialización y recorrido.
    \item \label{itm:8.3} Aplicaciones básicas.
  \end{enumerate}

  \item \textbf{Entrada y salida de archivos}

  \begin{enumerate}
    \item \label{itm:9.1} Concepto y organización de archivos.
    \item \label{itm:9.2} Lectura y escritura de archivos de texto.
  \end{enumerate}

  \item \textbf{Análisis básico de algoritmos}

  \begin{enumerate}
    \item \label{itm:10.1} Concepto de eficiencia.
    \item \label{itm:10.2} Complejidad temporal y espacial.
    \item \label{itm:10.3} Notación O grande.
  \end{enumerate}

  \item \textbf{Introducción a la teoría de grafos}

  \begin{enumerate}
    \item \label{itm:11.1} Concepto de grafo y aplicaciones.
    \item \label{itm:11.2} Vértices, aristas y tipos de grafos.
    \item \label{itm:11.3} Representaciones básicas (listas y matrices de adyacencia).
  \end{enumerate}

  \item \textbf{Teoría de estructuras de datos avanzadas}

  \begin{enumerate}
    \item \label{itm:12.1} Concepto de estructuras de datos.
    \item \label{itm:12.2} Pilas, colas, árboles y grafos como modelos abstractos.
    \item \label{itm:12.3} Casos de uso y comparación conceptual.
  \end{enumerate}

\end{enumerate}
