\section{Evaluación}

La evaluación del curso se basará en un enfoque formativo y acumulativo.
Las actividades evaluativas se distribuyen de la siguiente manera:

\begin{itemize}
  \item \textbf{Tareas (35\%)}: Asignaciones individuales o en parejas que permiten profundizar en la solución de problemas programáticos. Tendrán una semana de tiempo para completarlas.

  \item \textbf{Prácticas evaluadas (15\%)}: Actividades desarrolladas durante las clases sincrónicas, donde el estudiantado debe implementar soluciones funcionales.

  \item \textbf{Quices (10\%)}: Evaluaciones breves aplicadas de manera programada, orientadas a verificar la comprensión de los temas recién abordados.

  \item \textbf{Examen (20\%)}: Prueba individual que integra y evalúa los conocimientos adquiridos, enfocándose tanto en teoría como en resolución de problemas mediante código.

  \item \textbf{Proyecto final (20\%)}: Desarrollo de una aplicación funcional que combine los contenidos del curso, incluyendo programación orientada a objetos, estructuras dinámicas y control de versiones, entre otros.

\end{itemize}

A continuación, se desglosan los contenidos a evaluar en cada una de las evaluaciones:

\begin{longtable}{p{2.5cm}p{5cm}}
  \hline
  \hline
  \textbf{Evaluación} & \textbf{Temas evaluados} \\
  \hline
  PE01 & \ref{itm:1.1}, \ref{itm:1.2} \\
  \hline
  T01 & \ref{itm:1.3}, \ref{itm:1.4} \\
  \hline
  Q01 & \ref{itm:1.5}, \ref{itm:1.6} \\
  \hline
  PE02 & \ref{itm:2.1} -- \ref{itm:2.3} \\
  \hline
  Examen & \ref{itm:1.1} -- \ref{itm:2.4} \\
  \hline
  T02 & \ref{itm:3.1} -- \ref{itm:3.3} \\
  \hline
  Q02 & \ref{itm:3.3} \\
  \hline
  PE03 & \ref{itm:4.1} -- \ref{itm:4.3} \\
  \hline
  T03 & \ref{itm:5.1} -- \ref{itm:5.4} \\
  \hline
  Q03 & \ref{itm:5.1} -- \ref{itm:5.6} \\
  \hline
  PE04 & \ref{itm:6.1} -- \ref{itm:6.4} \\
  \hline
  Avance & Avance del proyecto final \\
  \hline
  Proyecto & Presentación y entrega final\\
  \hline
  \hline
\end{longtable}
